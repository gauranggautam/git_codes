\documentclass[letterpaper,12pt]{article}

\usepackage{url}
\usepackage{hyperref}

\usepackage{fancyvrb}
\DefineShortVerb{\|}
\DefineVerbatimEnvironment{code}{Verbatim}{}
%\DefineVerbatimEnvironment{code}{BVerbatim}{}

\title{pyHegel user manual}
\author{Christian Lupien}

\begin{document}
\maketitle

\section{Introduction}
This manual describes pyHegel. This program is used to communicate with
various laboratory instruments using protocols such as GPIB or VISA. It 
allows for the automation of data acquisition.

The program is written in python and relies on using a python console to enter
the various commands. Python is a powerfull interpreted language, so pyHegel
can easily be extended by writting python code interactivelly, define functions
or calling python scripts.

There are 2 python series. The 2.x and 3.x series. pyHegel is currently written
for the 2.7 version. For a introduction to python, see the tutorial at 
\url{http://docs.python.org/tutorial/}
and the documentation at
\url{http://docs.python.org/}.

To extend the capabilities of python, many modules can be used. for pyHegel
we use the pyVisa module to interface to instruments using visa device drivers.
Visa libraries should already be installed. They come from National Instruments
or Agilent. They allow interfacing to instruments using a GPIB bus, a USB cable
or LAN, depending on the instrument capabilities.

Other interesting modules, and those that would be usefull to know better in 
order to better use pyHegel are: numpy, scipy, matplotlib and iPython. 
The numpy and scipy packages are used for numerical calculations. For
documentation see \url{http://docs.scipy.org/doc/} and for tutorials see
\url{http://www.scipy.org/Tentative_NumPy_Tutorial} and
\url{http://docs.scipy.org/doc/scipy/reference/tutorial/}. 
Numpy provides multi-dimensional array structures and fast calculations on them.
It works in a similar way as matlab, IDL, yorick, Mathematica or other
command line data analysis programs. The scipy module uses numpy to provide
many usefull algorithm, like filtering, probality densites, fitting ...
The matplotlib module allows plotting of data with a syntax similar to 
matlab (called pyplot) but also posseses an object interface.
For documentation see \url{http://matplotlib.sourceforge.net/}
and for a tutorial see \url{http://matplotlib.sourceforge.net/users/pyplot_tutorial.html}.

Finally the iPython module is the recommanded interface to use. The python 
console is quite basic. The iPython package as improved the basic console to
provide much better interactive environment. It has powerfull command line
editing capabilities (command recall, tab expansion, macros, automation,
automatic saving or in/out values ...).
For documentation see \url{http://ipython.org/documentation.html}.
Note that the ipython console can import the codes from the numpy and matplotlib.pylab
modules and gives you direct access to those functions.



\section{Installation}
In order for pyHegel to run, you first need to have python installed
with all the necessary modules. On linux that sould already be the case. On
windows, the easiest way to achieve this is to install the pythonxy
package. It contains all that is needed and more, like python editing enviroments
such as spyder, a lot of the documentations, the Qt libraries to make portable
graphical user interfaces. Just download the latest version from 
\url{http://code.google.com/p/pythonxy/} or take the one on 
the local network. Then install it. Make sure to install it for all users
and customize the install to include the pyVisa module.

Once pyhton is installed you need to obtain the latest version of pyHegel.
Currently the version of pyHegel are being maintained in a git repository.
Git is a distributed version control system. Therefore you should download
and install Git from \url{http://git-scm.com/} or again from the local network.

Then you use git to install the latest version. First you clone one of the
remote repository locally. You can do that with Git-GUI. Select clone and then
use \url{git://bender.physique.usherbrooke.ca/~lupien/pyHegel} and place the 
result somewhere locallay (recommended location is
 |C:\Codes\pyHegel|). This should download the latest version. If you later
want a newer want, get back into Git-GUI, reopen the correct local directory,
then in remote/fetch and select the proper location.
If you followed above it should be called origin. Otherwise it might be
lupien-git or something similar. Entries with ssh in the name will not work
since the require a password (they are used when submitting modified code).

You now have the latest version of the core. To run it, now start an ipython console.
You do that by running the python(x,y) program. It is a graphical interface that
allows you to start many of the tools within pythonxy. Using it you start one
of the consoles. Then at the command prompt within the console, you 
need to change directory to the location of the code and then execute it.
You do this has follows
\begin{code}
cd C:/Codes/pyHegel
run -i pyHegel
\end{code}

You can now start to use pyHegel. You probably will want to load an instrument
and start comminicating with it.
\section{Basic structure}
pyHegel is based on the legacy Hegel program which was written in C++ by Bertrand Reulet and has a similar syntax.
However, since it is executed from a python console, you also have access to all
the python code you want.

The code comes with the main pyHegel.py file which contains the main commands.
The |instrument| module contains all the basic instrument classes and the
specific classes for various laboratory equipment (multimeters, sources, ...).
the |traces| module handles plotting, and |local_config| provides a list of
preconfigured instruments to load. Finally the |acq_board_instrument| module
handles the control of the fast acquisiation cards.

Every instrument is represented by an object, which is created from a class definition
that describe that instrument model
by instantiating it (in python parlance), here using the proper
VISA or GPIB address of the particular instrument. In python, the various 
components of an object are accessed by adding a dot (.) to the object and
putting the attribute name, for example |obj.freq| accesses the attribute |freq| (which can be another object, or a value) of the object |obj|.

Whithin the ipython console, the various attribute availble can be listed with
the tab completion. Just add the dot and press tab and a list of possibilites
will be printed. Usually, attributes starting with an underscore (|_|), are
reserved for internal used. In addition, to get information about any object
in ipython (including pyHegel instruments and devices) just precede or follow
the object with a question mark (?). For example
\begin{code}
sweep?
?sweep
\end{code}
both give the online documentation of sweep.
If you double the question mark (??) you obtain the python source code for
the object.

In ipyhton, when you call a function that returns a value on the console without
assignin it to a variable, the result is displayed, and saved in a buffer.
Like in mathematica, you can refer to the last result using the underscore (|_|).
Earlier results are referred as |_n| where any is the number of the entry.
Assignement are done with the equal sign |newvar=function()|.


Within an instrument multiple devices are present. These represent various
things that the instrument can do. For a generator, they could represent the
frequency and the amplitude. In genreral a device will allow to be changed with
a |set| command and to be read with a |get| command, however, some devices might
only do one of those. Every device also has a cache, accessed with |getcache|
that is the last value used in either get/set. Reading the cache skips
the communication with the device and is therefore faster and has no side effects.

Both |set| and |get| can accept optional arguments, depending on the device.
For example it could be the trace number, or a filename. In all those cases,
it is a python keyword argument and it has a default value. There are no 
other positional arguments. For exemple |scope.trace.get(m=1)| could be
a valid call.

Every device that can be |set| also has a |check| function that verifies
if the parameters are valid. It is automatically called on all set.

The functions available for a device can be called directly. To get the frequency from a generator might be done like this: |gen.freq.get()|
The parenthesis are needed because get is a python function.
Every device also has a few shortcuts. Calling the device itself is the
same has obtaining the cache. So the following are the same
\begin{code}
gen.freq.getcache()
gen.freq()
\end{code}
Additionally, calling a device with a value is the same as set so
\begin{code}
gen.freq.set(5e7)
gen.freq(5e7)
\end{code}
are the same.

Besides devices, instruments can also have commandes that are called directly.
For example the acquisistion cards has |set_histogram| function that is used
to turn on the histogram mode of the card. |set_histogram| is not a device and
you cannot perform |set| or |get| on it.

\section{Commands}
All the following are python functions and therefore require parentesis.
However, when entering commands on the ipython console interactivelly, the
parentesis around the arguments can be skipped and ipython will add them 
automatically. This will not work when no argument are present, then you
need to add them yourself or use the ipython trick of preceding the function
name by a slash (|/|). These shortcut will not work for script files,
unless the script is designated and ipython script by having the extension
.ipy. Regular python code/script files have extension .py.

\begin{code}
load
find_all_instruments()
iprint(device)
ilist()
dlist()

set
get
getasync
move
copy
sleep(time)
wait

spy
record
trace

scope
check
checkmode
batch
also execfile, run or run -i
Execute system command
os.system or !
pass : to skip
comments in python are #

make_dir

task
top
kill

var???

reset_pyHegel()
collect_garbage()

clock
sweep
\end{code}
The sweep device have an additional syntax available you can assign to them:
\begin{code}
set(sweep.path,'Somewhere/on/disk')
sweep.path('Somewhere/on/disk')
sweep.path='Somewhere/on/disk'
\end{code}
All do the same thing.


Can alias an instrument or device by assigning it to another variable
|newname=instrument1|, |newname = instrument1.device1|

You can forget about a loaded instrument by deleting the the python object:
|del instrument1|. However, if the object was assigned to other variables (including the ipython numbered results) the object will not get deleted. Deletion
in python only occures when the last reference to the object disappeared.
This might be delayed by the garbage collector and not be possible because
of circular references (there should not be circular refenrences in instruments
and devices however). Left over devices are usually not a big problem since they
don't consume much memory and closing a device does not usually communicate
with it.

Of course all numpy and matplotlib functions should already be loaded so 
you can use:
plot
random
arange
linspace
hist
etc...

\end{document}

